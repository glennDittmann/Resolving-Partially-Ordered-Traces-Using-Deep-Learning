\documentclass[a4paper,12pt]{scrartcl}
\usepackage{aufgaben}
\usepackage{natbib}
\usepackage{graphicx}
\usepackage{amsmath}
\usepackage{url}
\usepackage{cancel}

% ------------------------- DOCUMENT ------------------------------------------

\docdate{put date here} %== due date
\docnum{1}
\doctitle{put topic here}
%\docpoints{BA: 25, MA/D: 43}
%\doccomment{Please solve the following exercise tasks. Please post notes on your solutions into the Forum, so that we can use these notes as a starting point in the discussion of the solutions.
%}



%\include{commonDefs}

% ------------------------- DOCUMENT -------------------------
\begin{document}

\maketitle

\noindent Consider the initial example $\mathcal{T}_{uncertain} = (\{u\}, \{v,w,x\}, \{y,z\})$ \\
Then $\mathcal{A} = \{u, v, w, x, y, z\}$\\
With possible resolutions being: 
\begin{align*}
	\{u\}     &: \text{u} \\
	\{v,w,x\} &: \text{vwx, vxw, wvx, wxv, xvw, xwv} \\
	\{y,z\}   &: \text{yz, zy}
\end{align*}

\noindent For an uncertain set of size $s$ there are $s!$ possible resolutions. \\
For an activity space of size $a$ and there are $a^l$ possible resolutions of length $l$. Therefore the size of a one-hot vector encdoing for all possible resolutions up to some length $K$ is $\sum_{k=1}^K a^k$.


\section*{LSTM}
input sequence length == output sequence length

	\paragraph*{Basic encoding with repeated uncertain sets}
		
	
	\paragraph*{Sets/Sequence with one $1$ per uncertain set}
	
		
	\paragraph*{Sets/Sequence with multiple $1$s per uncertain set}

	
\section*{Seq2Seq}
input sequence length $<>$ output sequence length

	\paragraph*{Basic encoding (no need for repitition)}
	
	
	\paragraph*{Sets/Sequence with one $1$ per uncertain set}
	
	
	\paragraph*{Sets/Sequence with multiple $1$s per uncertain set}


\section*{To Consider}
	\begin{itemize}
		\item The output MUST be a resolution
		\item So, if the output with the highest probability is not a resolution, get the resolution with the highest probability
		\begin{itemize}
			\item Is that easy to achieve? Try to get the activation information on the output layer!
			\item Note: this can be handled on the level of uncertain sets. Once the sequence for an uncertain set has been predicted it is clear whether this sequence is part of a valid resolution
		\end{itemize} 
	\end{itemize}

\end{document}
% ------------------------- DOCUMENT -------------------------
