\documentclass[
	a4paper,
	pagesize,
	pdftex,
	12pt,
	%twoside, % + BCOR darunter: für doppelseitigen Druck aktivieren, sonst beide deaktivieren
	%BCOR=5mm, % Dicke der Bindung berücksichtigen (Copyshop fragen, wie viel das ist)
	ngerman,
	fleqn,
	final,
	]{scrartcl}
\usepackage{ucs}
\usepackage[utf8x]{inputenc} % Eingabekodierung: UTF-8
\usepackage[T1]{fontenc} % ordentliche Trennung
\usepackage[british]{babel}
\usepackage{lmodern} % ordentliche Schriften
\usepackage[unicode=true]{hyperref}
\usepackage{setspace,graphicx,tikz,tabularx} % für Elemente der Titelseite
\usepackage[draft=false,babel,tracking=true,kerning=true,spacing=true]{microtype} % optischer Randausgleich etc.
%\usepackage{natbib}
\usepackage[ddmmyyyy]{datetime}
\usepackage{amsthm}
\usepackage{subfig}
\usepackage{graphicx}
\usepackage{amssymb}
\usepackage{mathtools}

\theoremstyle{plain}
\newtheorem{thm}{Theorem}

\theoremstyle{definition}
\newtheorem{defn}[thm]{Definition}

\renewcommand{\labelitemi}{$\diamond$}

\begin{document}

% Beispielhafte Nutzung der Vorlage für die Titelseite (bitte anpassen):
% LaTeX-Vorlage für die Titelseite und Selbständigkeitserklärung einer Abschlussarbeit
% basierend auf der vorigen Institutsvorlage des Instituts für Informatik
% sowie der Vorlage für Promotionsarbeiten.
%
% erweitert: 2014-06-12 Dennis Schneider <dschneid@informatik.hu-berlin.de>

% gepunktete Linie unter Objekt:
\newcommand{\TitelPunkte}[1]{%
  \tikz[baseline=(todotted.base)]{
    \node[inner sep=1pt,outer sep=0pt] (todotted) {#1};
    \draw[dotted] (todotted.south west) -- (todotted.south east);
  }%
}%

% gepunktete Linie mit gegebener Länge:
\newcommand{\TitelPunktLinie}[1]{\TitelPunkte{\makebox[#1][l]{}}}

\makeatletter

\newcommand*{\@titelTitel}{Titel der Arbeit}
\newcommand{\titel}[1]{\renewcommand*{\@titelTitel}{#1}} % Titel der Arbeit
\newcommand*{\@titelArbeit}{Arbeitstyp}
\newcommand{\typ}[1]{\renewcommand*{\@titelArbeit}{#1}} % Typ der Arbeit
\newcommand*{\@titelGrad}{akademischer Grad}
\newcommand{\grad}[1]{\renewcommand*{\@titelGrad}{#1}} % Akademischer Grad
\newcommand*{\@titelAutor}{Autor}
\newcommand{\autor}[1]{\renewcommand*{\@titelAutor}{#1}} % Autor der Arbeit
\newcommand*{\@titelGeburtsdatum}{\TitelPunktLinie{2cm}}
\newcommand{\gebdatum}[1]{\renewcommand*{\@titelGeburtsdatum}{#1}} % Geburtsdatum des Autors
\newcommand*{\@titelGeburtsort}{\TitelPunktLinie{5cm}}
\newcommand{\gebort}[1]{\renewcommand*{\@titelGeburtsort}{#1}} % Geburtsort des Autors
\newcommand*{\@titelGutachterA}{\TitelPunktLinie{5cm}}
\newcommand*{\@titelGutachterB}{\TitelPunktLinie{5cm}}
\newcommand{\gutachter}[2]{\renewcommand*{\@titelGutachterA}{#1}\renewcommand*{\@titelGutachterB}{#2}} % Erst- und Zweitgutachter
\newcommand*{\@titelEinreichungsdatum}{\TitelPunktLinie{3cm}} % Datum der Einreichung, wird nicht vom Studenten ausgefüllt
\newcommand*{\@titelVerteidigungsdatum}{} % Verteidigungstext, wird nicht vom Studenten ausgefüllt
\newcommand{\mitverteidigung}{\renewcommand*{\@titelVerteidigungsdatum}{verteidigt am: \,\,\TitelPunktLinie{3cm}}} % Verteidigungsplatzhalter erzeugen
\newcommand*{\@wastwoside}{}

% Titelseite erzeugen:
\newcommand{\makeTitel}{%
	% Speichere, ob doppelseitiges Layout gewählt wurde:
\if@twoside%
	\renewcommand*{\@wastwoside}{twoside}
\else
	\renewcommand*{\@wastwoside}{twoside=false}
\fi
	\KOMAoptions{twoside = false}% Erzwinge einseitiges Layout (erzeugt eine Warnung)

	\begin{titlepage}
		% Ändern der Einrückungen
		\newlength{\parindentbak} \setlength{\parindentbak}{\parindent}
		\newlength{\parskipbak} \setlength{\parskipbak}{\parskip}
		\setlength{\parindent}{0pt}
		\setlength{\parskip}{\baselineskip}

		\thispagestyle{empty}

		\begin{minipage}[c][3cm][c]{12cm}
			\textsc{%
				% optischer Randausgleich per Hand:
				\hspace{-0.4mm}\textls*[68]{\Large Humboldt-Universität zu Berlin}\\
				\normalsize \textls*[45]{
					Faculty of Mathematics and Natural Sciences\\
					Department of Computer Science \\
					Chair of Databases and Information Systems
				}
			}
		\end{minipage}
		\hfill


		% Also wenn schon serifenlose Schriften (Titel), dann ganz oder gar nicht
		\sffamily

		\vfill

		\begin{center}
		\begin{doublespace}
			\vspace{\baselineskip}
			{\LARGE \textbf{\@titelTitel}}\\
			%\vspace{1\baselineskip}
			{\Large
				\@titelArbeit\\
				what to write here?\\
				\@titelGrad
				\vspace{\baselineskip}
			}
		\end{doublespace}
		\end{center}

		\vfill
\newcolumntype{L}{>{\raggedright\arraybackslash}X}
		{\large \raggedleft
			\begin{tabularx}{\textwidth}{l@{\,\,\raggedright~}L} % verbreiterter Abstand zwischen Feldern wurde gewünscht
				submitted by: & \@titelAutor\\
				born on: & {\@titelGeburtsdatum}\\
				born in: & \@titelGeburtsort
				\vspace{0.5\baselineskip}\\
				reviewer: & \@titelGutachterA \\
					& \@titelGutachterB
				\vspace{0.5\baselineskip}\\
				submitted on: & \@titelEinreichungsdatum \hfill \@titelVerteidigungsdatum
			\end{tabularx}}
			\vspace{-1\baselineskip}\\\phantom{x} % Übler Hack, um eine Warnung wg. einer zu leeren hbox zu verhindern
		% Wiederherstellen der Einrückung
		\setlength{\parindent}{\parindentbak}
		\setlength{\parskip}{\parskipbak}
	\end{titlepage}

	% Aufräumen:
	\let\@titelTitel\undefined
	\let\titel\undefined
	\let\@titelArbeit\undefined
	\let\typ\undefined
	\let\@titelGrad\undefined
	\let\grad\undefined
	\let\@titelAutor\undefined
	\let\autor\undefined
	\let\@titelGeburtsdatum\undefined
	\let\gebdatum\undefined
	\let\@titelGeburtsort\undefined
	\let\gebort\undefined
	\let\@titelGutachterA\undefined
	\let\@titelGutachterB\undefined
	\let\gutachter\undefined
	\let\@titelEinreichungsdatum\undefined
	\let\einreichungsdatum\undefined
	\let\@titelVerteidigungsdatum\undefined
	\let\verteidigungsdatum\undefined

	\KOMAoptions{\@wastwoside}% Stelle alten Modus (ein-/doppelseitig) wieder her
	\let\@wastwoside\undefined
	\cleardoublepage % ganzes Blatt für die Titelseite
}

% Als Allerallerletztes kommt Selbständigkeitserklärung:
% Aufruf mit dem Datum in deutscher und englischer Form
\newcommand{\selbstaendigkeitserklaerung}[1]{%
	\cleardoublepage% Wieder auf eine eigene Doppelseite
	{\parindent0cm
		\subsection*{Statement of Authorship}
		This thesis, in same or similar form, has not been submitted to any institution yet. \\
		\\
		I especially confirm that without any exception, I have fully cited all sources referring to
		direct quotations of other authors' statements or tables, graphs, quotations as well as all indirect 
		quotations or modified tables, graphics and citations from the internet. \\
		\\
		I am aware that any breach against these rules is considered as plagiarism and wil be punished according
		to the general university regulations (Allgemeine Satzung zur Regelung von Zulassung, Studium und Prüfung Humboldt-Universität zu Berlin, ZSP-HU).
		\vspace{3\baselineskip}

		{\raggedright Berlin, #1 \hfill \TitelPunktLinie{8cm}\\}
%		\vspace{3\baselineskip}
%
% 		\selectlanguage{english}
% 		\subsection*{Statement of authorship}
% 		Hier würde die englische Selbständigkeitserklärung folgen, falls gewünscht. Doch es fehlt eine akzeptable Übersetzung.
% 		\vspace{3\baselineskip}
%
% 		Berlin, #2 \hfill \TitelPunktLinie{6cm}
	}
}%

\makeatother

\titel{Neural Networks for Partially Ordered Trace Resolution} % Titel der Arbeit
\typ{Bachelorarbeit} % Typ der Arbeit:  Diplomarbeit, Masterarbeit, Bachelorarbeit
\grad{Bachelor of Science (B. Sc.)} % erreichter Akademischer Grad
% z.B.: Master of Science (M. Sc.), Master of Education (M. Ed.), Bachelor of Science (B. Sc.), Bachelor of Arts (B. A.), Diplominformatikerin
\autor{Glenn Dittmann} % Autor der Arbeit, mit Vor- und Nachname
\gebdatum{13.06.1993} % Geburtsdatum des Autors
\gebort{Berlin} % Geburtsort des Autors
\gutachter{Prof. Dr. Matthias Weidlich}{Dr. Wolfgang Koessler} % Erst- und Zweitgutachter der Arbeit
\mitverteidigung % entfernen, falls keine Verteidigung erfolgt
\makeTitel

%abstract

% Hier folgt die eigentliche Arbeit (bei doppelseitigem Druck auf einem neuen Blatt):
\tableofcontents
\newpage

\textsf{Purpose and scope of your entire report.} The purpose of your entire report is to make a 
\textbf{scientific argument using the scientific method}.A scientific argument always has the following steps that all must come in this order.

\begin{itemize}
	\item[SM1] \textbf{Explicate the assumptions and state of the art} on which you are going to conduct your research to investigate your research problem / test the hypothesis.
	\item[SM2] Clearly and precisely \textbf{formulate a research problem or hypothesis}.
	\item[SM3] \textbf{Describe the (research) method} that you followed to investigate the problem / to test the hypothesis in a way that \textbf{allows someone else to reproduce your steps}. The method must include steps and criteria for evaluating whether you answered your question successfully or not.
	\item[SM4] \textbf{Provide execution details} on how you followed the method in the given, specific situation.
	\item[SM5] \textbf{Report your results} by describing and summarizing your measurements. You must not interpret your results.
	\item[SM6] \textbf{Now interpret your results} by contextualizing your measurements and drawing conclusion that lead to answering your research problem or defining further follow-up research problems.
	
\end{itemize}

%%% ===============================================================================
\section{Introduction}\label{sec:introduction}
%%% ===============================================================================
	\textsf{Purpose and scope of Section \ref{sec:introduction}}. The introduction is a summary of your work and your scientific argument that shall be understandable to anyone in your scientific field, e.g., anyone in Data Science. A reader must be able to comprehend the problem, method, relevant execution details, results, and their interpretation by reading the introduction and the introduction alone.
	Section~\ref{sec:introduction::topic} introduces the general topic of your research
	Section~\ref{sec:introduction::state-of-art} discusses the state of the art and identifies a research.
	Section~\ref{sec:introduction::research-question} then states the research problem to investigate.
	Section~\ref{sec:introduction::method} explains the research method that was followed, possibly with execution details.
	Section~\ref{sec:introduction::results} then presents the results and their interpretation. Only if a reader thinks they are not convinced or they need more details to reproduce your study, they shall have to read further. The individual chapters and sections provide the details for each of the steps in your scientific argument.
	
	You usually write the introduction chapter \emph{after} you wrote all other chapters, but you should keep on making notes for each of the subsections as you write the later chapters.
	
	\subsection{Context and Topic (SM1)}\label{sec:introduction::topic}
	
	\subsection{State of the Art (SM1)}\label{sec:introduction::state-of-art}
	
	\subsection{Research Question (SM2)}\label{sec:introduction::research-question}
	
	\subsection{Method or Approach (SM3, SM4)}\label{sec:introduction::method}
	
	\subsection{Findings (SM5, SM6)}\label{sec:introduction::results}

%%% ===============================================================================
\section{Background}
%%% ===============================================================================
	\subsection{Preliminaries}
		Process Mining -> Conformance Checking \\
		\begin{defn}{Activity}
			Activities are ... captured in the set of all activities.
		\end{defn}
		\begin{defn}{Event}
			An event is... mapping from events to activities; events are unique in a log
		\end{defn}
		\begin{defn}{Trace}
			A trace is...
		\end{defn}
		\begin{defn}{Log}
			A log is...
		\end{defn}
		Based on the timestamps associated with events in a given log, we define a partial ordering of events for a certain trace. Thus an event e1 < e2, if the timestamp of e1, i.e. time(e1), is smaller than/happens before, the timestamp of e2, i.e time(e2). Otherwise the timestamps of e1 and e2 are the same, leading to e1 = e2, i.e. time(e1) = time(e2).
		\begin{defn}{Uncertain Trace}
			A trace is called uncertain if there is a trace for which for any two events ei, ej time(ei) = time(ej)
		\end{defn}
		\begin{defn}{Certain Trace}
			A trace is called certain if for all events e1 to en for all traces, ei < ei+1.
		\end{defn}
		Based on this, if all traces in a log are certain, we call it a certain log and an uncertain log otherwise. \\
		Event-Name, Event-Alias, Event-Properties \\
		Neural Network \\
		Partial Order, Total Order \\ 
	
	\subsection{Related Work}
	The field of conformance checking and process mining is very broad, so a lot of research has been done there up to today. Furthermore the field of machine learning has reached another peak of high interest in business applications, as well as media, teaching and research. 
	The particular problem of solving partially ordered event logs from real-life process applications though, has been investigated in a fairly small amount compared to the above. 
	Different techniques have been used to solve the problem differently and machine learning was only used once for predicting information of event logs. \\
	To my best knowledge and also stated in previous work \cite{lu2014conformance} so far there has been little research done addressing the problem of only partially ordered event logs.\\
	M. de Leoni et al. presented a technique to abstract the problem of aligning partially ordered traces into a PDDL-encoded planning problem. This approach will either report, that there exists no solution, i.e. optimal alignment, for an explicit trace and petri net. Or it will, in finite time, find an optimal alignment for a trace and petri net, whether or not the trace is sequential, i.e. totally ordered or a trace containing concurrent events, i.e. partially ordered. (note that by definition every totally ordered trace it also a partially ordered trace).\cite{de2018aligning} \\
	Van der Aalst et al. took another approach in defining partially ordered traces and from those compute partially ordered alignments, with the aim to provide a model that can express concurrently running events and from there getting insight in the meaning of those. \cite{lu2014conformanceShort} They researched the usefulness of those partially-ordered alignments with case studies relying on real-world data from a Dutch hospital. \cite{lu2014conformance} \\
	In \cite{tax2017predictive} Niek Tax et. al. introduce an approach to tackle three question not yet visited in the field of process mining. They use RNN's, LSTM's specifically, to answer three questions 1),2),3). However they let the order certainity of traces unvisited and thus, while exploring an answer for the next activity, omit the information of certain parts of a trace already being order / they only give answer on how future traces could end most possibly. \\
	Weidlich et. al. have sought three algorithmic approaches for resolving partially ordered traces in giving a probability distribution over all possible resolutions and from there on efficiently compute the conformance of partially ordered traces with a given process model.\cite{self} \\
	As of my best knowledge no research has yet been done, to resolve partial ordering of traces in conformance checking / process mining. We expect to find valuable solutions for resolving the partial ordering of events happening at the same time, exploiting the field of deep learning, neural networks respectively. \\
	Trying to cite the books read so far: \\
	Process Mining \cite{AalstWilvander2016Pm:d} \\ 
	Conformance Checking \cite{carmona2018conformance} \\
	Deep Learning \cite{Goodfellow-et-al-2016}\\
	Hands-on ML \cite{geron2019hands}
	
	
%%% ===============================================================================
\section{Problem Exposition (optional)}
%%% ===============================================================================
	
	\subsection{Context / Business Understanding (SM1)}
	
	\subsection{Data Understanding (SM1)}
	
	\subsection{Detailed Research Questions (SM2)}
	
	\subsection{Detailed Method (SM3)}

%%% ===============================================================================
\section{First Real Chapter addressing first Research Problem}
%%% ===============================================================================

	\subsection{First Sub-Problem}
	
	\subsection{Second Sub-Problem}

%%% ===============================================================================
\section{Second Real Chapter}
%%% ===============================================================================

%%% ===============================================================================
\section{Evaluation}
%%% ===============================================================================

	\subsection{Objective (SM2)}
	
	\subsection{Setup (SM3)}
	
	\subsection{Execution (SM4)}
	
	\subsection{Results (SM5)}
	
	\subsection{Discussion (SM6)}

%%% ===============================================================================
\section{Conclusion}
%%% ===============================================================================
		Your conclusions are not just a factual summary of your work, but they position, interpret and defend your findings against the state of the art that you discussed in Sect.~\ref{sec:introduction::state-of-art}. You specifically outline which concrete findings or methodological contributions advance our knowledge towards the general objective you introduced in Sect.~\ref{sec:introduction::topic}. Objectively discuss which parts you solved and in which parts you failed. \\
		You should explicitly discuss limitations and shortcomings of your work and detail what kind of future studies are needed to overcome these limitations. Be specific in the sense that your arguments for future work should be based on concrete findings and insights you obtained in your report.

%%% ===============================================================================
%%% Bibliography
%%% ===============================================================================		
\bibliographystyle{alpha}
\bibliography{bibliography}



% Erzeugen der Selbständigkeitserklärung auf einem neuen Blatt:
\selbstaendigkeitserklaerung{\today}

\end{document}
